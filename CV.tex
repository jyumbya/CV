%%%%%%%%%%%%%%%%%%%%%%%%%%%%%%%%%%%%%%%%%
% This document is based on a template available at
% http://www.LaTeXTemplates.com
%
% Original author:
% Trey Hunner (http://www.treyhunner.com/)
%%%%%%%%%%%%%%%%%%%%%%%%%%%%%%%%%%%%%%%%%

%----------------------------------------------------------------------------------------
%	PACKAGES AND OTHER DOCUMENT CONFIGURATIONS
%----------------------------------------------------------------------------------------

\documentclass{resume} % Use the custom resume.cls style

\usepackage[left=0.75in,top=0.6in,right=0.75in,bottom=0.6in]{geometry} % Document margins
\usepackage{fancyhdr}
\usepackage{lastpage}
\usepackage{amssymb}
\usepackage{url}

%\input{officialheader.tex} %If you want this file to compile to my CV, comment out this line and uncomment the next three
\name{John Yumbya Mutua} 
\address{{jyumbya@yahoo.com} \\ \url{jyumbya.github.io}  \\ @jyumbya } 
\address{PO Box 45344 \\ Nairobi \\ Kenya} 

\pagestyle{fancy}
\fancyhf{}
\renewcommand{\headrulewidth}{0pt}
\cfoot{jyumbya, page \thepage\ of \pageref{LastPage}}
\begin{document}


\begin{rSection}{Research interests}
GIS, remote sensing, spatial statistics, data visualization, reproducible research
\end{rSection}

%----------------------------------------------------------------------------------------
%	EDUCATION
%----------------------------------------------------------------------------------------

\begin{rSection}{Education}
{\bf Master of Climate Change Adaptation} \hfill Continuing\\ 
{\bf University of Nairobi} \hfill Nairobi, Kenya \\
{Mapping heat stress risk for dairy cattle under current and future climatic conditions in Uganda} \smallskip 

{\bf B.A., Geography} \hfill  2009 \\
{\bf Moi University} \hfill Eldoret, Kenya\smallskip 
\end{rSection}

%----------------------------------------------------------------------------------------
%	PAPERS
%----------------------------------------------------------------------------------------

\begin{rSection}{Peer-reviewed Publications}

Nijbroek, R., Piikki, K., S\"oderstr\"om, M., Kempen, B., Turner, K., Hengari, S., and {\bf \em{Mutua, J.}} (2018). { \bf Soil Organic Carbon Baselines for Land Degradation Neutrality: Map Accuracy and Cost Tradeoffs with Respect to Complexity in Otjozondjupa, Namibia. } {\em Sustainability }, 10(5), 1610.

Hamerlynck, O., Zeine, S. M., {\bf \em{Mutua, J.,}} Mukhwana, L., and Y\'ena, M. (2016). { \bf Reflooding the Faguibine floodplain system, northern Mali: potential benefits and challenges. } {\em African Journal of Aquatic Science }, 41(1), 109-117.

{\bf \em{Yumbya, J.,}} Vaate, M. B. De, Kiambi, D., Kebebew, F., and Rao, K. 2014. { \bf Geographic Information Systems for assessment of climate change effects on teff in Ethiopia. } {\em African Crop Science Journal }, 22, 847 - 858

\end{rSection}

%----------------------------------------------------------------------------------------
%	RESEARCH
%----------------------------------------------------------------------------------------

\begin{rSection}{Research and Consulting}
\begin{pSubsection}{International Centre for Tropical Agriculture, }{Nairobi, Kenya}{2015-Present}{Geospatial Analyst}
{My work revolves around creating reproducible workflows for analyzing complex spatial data, with appropriate analytical methods, informed by my unique combination of skills, experience, and understanding of research themes - climate change, food security and agriculture.}
\end{pSubsection}

\begin{pSubsection}{African Biodiversity Conservation and Innovations Centre, }{Nairobi, Kenya}{October 2010-March 2015}{GIS Analyst}
{Designed and implemented models for mapping commodity suitability in East Africa and implemented training and capacity building programs in GIS and remote sensing.}
\end{pSubsection}


\begin{pSubsection}{International Centre for Tropical Agriculture, }{Cali, Colombia}{February 2012-June 2012}{Visiting Scientist}
{Developed an R algorithm for mapping and identifying priority rice wild relatives species for conservation.}
\end{pSubsection}

\begin{pSubsection}{Institute of Research for Development, }{Tombouctou, Mali}{April 2010-September 2010}{GIS/Remote Sensing Consultant}
{Spectral analysis of Multispectral Landsat 7 Enhanced Thematic Mapper Plus (ETM+) and aerial imagery; Created geospatial datasets for use in ecosystem services mapping projects in Kenya and Mali.}
\end{pSubsection}


\begin{pSubsection}{Value Addition and Cottage Industry Development in Africa, }{Nairobi, Kenya}{January 2010-September 2010}{GIS Analyst}
{Maintained the organization's enterprise geodatabase and ensured constant updating; Performed GIS data cleaning; Developed and implemented geostatistical modeling and analysis.}
\end{pSubsection}


\begin{pSubsection}{Kenya Electricity Generating Company, }{Nairobi, Kenya}{September 2009-January 2010}{Intern}
{Hydrological modeling and forecasting for hydroelectric power production; Maintained the organization's enterprise geodatabase and ensured constant updating.}
\end{pSubsection}


\begin{pSubsection}{Department of Resource Survey and Remote Sensing, }{Nairobi, Kenya}{June 2009-August 2009}{Intern}
{Designed and produced high-resolution well-designed map layouts; Maintained the organization's enterprise geodatabase and ensured constant updating.}
\end{pSubsection}


\begin{pSubsection}{Kenya Meteorological Department}{Nairobi, Kenya}{June 2008-August 2008}{Intern}
{Designed and implemented a spatial model for mapping daily rainfall distribution in Kenya; Modeled rainfall-runoff modeling using Galway Flow Forecasting System.}
\end{pSubsection}
\end{rSection}

% \clearpage

%----------------------------------------------------------------------------------------
%	SHORT COURSES
%----------------------------------------------------------------------------------------

\begin{rSection}{Short courses and workshops led}

\begin{sSubsection}{R for developing LDN indicators}{}{June 2017}{ Faculty of Agriculture and Natural Resources, University of Namibia, }{Windhoek, Namibia}
\end{sSubsection}
\end{rSection}

%----------------------------------------------------------------------------------------
%	INVITED TALKS
%----------------------------------------------------------------------------------------

\begin{rSection}{Invited talks}

\begin{sSubsection}{Geospatial Innovation at CIAT: Today and a View into the future}{}{November 2017}{Kenyatta University, GIS Day}{Nairobi, Kenya}
\end{sSubsection}
\end{rSection}

%----------------------------------------------------------------------------------------
%	PRESENTATIONS
%----------------------------------------------------------------------------------------
\begin{rSection}{Conference Presentations}

\begin{sSubsection}{Bush encroachment mapping in Otjozondjupa region, Namibia}{ }{August 2018}{ FOSS4G 2018}{Dar es salaam, Tanzania}
\end{sSubsection}
\end{rSection}

%----------------------------------------------------------------------------------------
%	POSTERS
%----------------------------------------------------------------------------------------
\begin{rSection}{Poster presentations}

\begin{sSubsection}{Mapping the suitability of tropical forages - now and in the future -}{}{September 2018}{Tropentag}{Ghent, Belgium}
\end{sSubsection}

\begin{sSubsection}{CIAT Soil Carbon Sequestration Research}{}{November 2017}{Soil Carbon Sequestration: Supporting NDCs and donor action}{ZEF Center for Development Research, University of Bonn}
\end{sSubsection}

\begin{sSubsection}{Development of a SOC Baseline: Experiences from Otjozondjupa, Namibia.}{}{March 2017}{Global Symposium on Soil Organic Carbon}{Rome, Italy}
\end{sSubsection}

\begin{sSubsection}{Spatial Targeting of Agricultural Intensification Investments}{}{March 2017}{Linking Household Surveys with Spatial Data Workshop}{Arusha, Tanzania}
\end{sSubsection}
\end{rSection}


%----------------------------------------------------------------------------------------
%	SKILLS
%----------------------------------------------------------------------------------------


\begin{rSection}{Skills}

\begin{tabular}{ @{} >{\bfseries}l @{\hspace{6ex}} l }
Software & Expert: ArcGIS Suite, QGIS, ERDAS Imagine suite, ENVI, Google Earth Engine \\

Languages & Expert: R, \LaTeX, HTML, Markdown, XLSForm, git \\
& Intermediate: Python, MATLAB, JavaScript \\

Social Media & Twitter: \textbf{@JohnYumbya}, 1,800+ tweets, 176+ followers \\
& Blogs: maintainer, \url{https://jyumbya.github.io/} \\

\end{tabular}

{\bf Open source contributions:} \\
SurfaceTortoise - Find Optimal Sampling Locations Based on Spatial Covariate(s): \url{https://github.com/cran/SurfaceTortoise} \\
mapsRinteractive: Local Adaptation and Evaluation of Digital Soil Maps: \url{https://github.com/cran/mapsRinteractive} \\
Targeting Tools: \url{https://github.com/CIAT/targeting_tools_10_3} \\
CLEANED-RX: \url{https://github.com/CIAT/cleaned-rx} \\
% R packages:
Many more projects hosted on GitHub, \url{https://github.com/jyumbya} \\
\end{rSection}


%----------------------------------------------------------------------------------------
%	MEMBERSHIPS
%----------------------------------------------------------------------------------------

\begin{rSection}{Memberships in Professional Societies}
\begin{esSubsection}{Geospatial Society of Kenya }{(GSK)}{2017-present}{}{}
\end{esSubsection}

\begin{esSubsection}{Society for Conservation GIS }{(SCGIS)}{2010-present}{}{}
\end{esSubsection}
\end{rSection}



%----------------------------------------------------------------------------------------
%	References
%----------------------------------------------------------------------------------------
% \clearpage
% \input{references.tex} %Comment out to get this to compile


\begin{rSection}{References}
\begin{esSubsection}
{Dr. Dionysious K Kiambi}\\
{Dean, Graduate School}\\
{Pan Africa Christian University}\\
{P.O Box 56875 - 00200, Nairobi, Kenya}\\
{dionysious.kiambi@pacuniversity.ac.ke}
\end{esSubsection}

\begin{esSubsection}
{William  K.Kiplagat}\\
{Head, Geography Department}\\
{Moi University}\\
{P.O. Box 3900 - 30300, Eldoret, Kenya}\\
{williamkiplagat@yahoo.com}
\end{esSubsection}

\begin{esSubsection}
{Dr. James M Kinyanjui}\\
{Natural Resource Scientist}\\
{}\\
{Department of Resource Surveys and Remote Sensing}\\
{P.O. Box 47146 - 00100, Nairobi, Kenya}\\
{mwangikinyanjui@gmail.com}
\end{esSubsection}

\end{rSection}
\end{document}
